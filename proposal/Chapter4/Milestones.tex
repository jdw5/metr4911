\chapter[Milestones]{Research Plan}

\label{Chap:Milestones}

\section{Milestones}
To track the development of the project, a set of milestones have been established with consideration of assessable dates. 
Table \ref{tab:milestones} below outlines the milestones for the work and what is required for each phase.
Emboldened text indicates an assessable milestone.


\begin{center}
    \small
    \begin{longtable}{p{0.25\linewidth}p{0.5\linewidth}p{0.15\linewidth}}
        \caption{Timeline of milestones.} \label{tab:milestones} \\
        \toprule
        Milestone & Description & Date \\
        \midrule
        \endfirsthead
        
        \caption{Timeline of milestones. (continued)} \\
        \toprule
        Milestone & Description & Date \\
        \midrule
        \endhead
        
        \bottomrule
        \multicolumn{3}{r}{\textit{Continued on next page}} \\
        \endfoot
        
        \bottomrule
        \endlastfoot
        
        \textbf{Project Proposal}	& Complete the project proposal and literature review.                                              & 21/3 \\
        Configure hardware			& Implement NEORV32 processor on FPGA, connect to external components through wishbone interface.   & 15/4 \\
        \textbf{Seminar}            & Present the current project status.                                                               & 6/5 \\
        Image detection algorithm   & Select and implement image detection algorithm in hardware.                                       & 12/8 \\
        VGA driver                  & Develop driver and interface to display processed image on VGA monitor.                           & 9/9 \\
        Benchmark                   & Benchmark system performance against related works.                                               & 4/10 \\
        \textbf{Demo}               & Demonstrate the complete project.                                                                 & 18/10 \\
        \textbf{Thesis}             & Document and write project results.                                                               & 4/11 \\
    \end{longtable}
\end{center}

\nopagebreak

The timeline was developed based on the expected time to completion for each task. A more detailed overview of the non-assessable tasks required for each milestone is provided in the following sections.

\subsection{Configure Hardware}
This task has been allocated three weeks of work, estimated at 20 hours of work. 
It entails the implementation of the NEORV32 processor on the FPGA, and the connection of the processor to the external components through the wishbone interface.

\subsection{Image Detection Algorithm}
This task has been allocated three months of work, estimated at 100 hours of work.
It requires the selection, implementation, and testing of the image detection algorithm in hardware.
Furthermore, it must additionally interface with the NEORV32 processor, and be able to be controlled by the processor.
As it is the core of the project, it has been allocated the most time.

\subsection{VGA Driver}
This task has been allocated four weeks of work, estimated at 30 hours of work.
The task requires the development of a driver and interface to display the processed image on a VGA monitor for the selected algorithm.
It is expected to be a relatively simple task, but has been allocated additional time for any unforeseen issues.

\subsection{Benchmark}
This task has been allocated four weeks of work, estimated at 30 hours of work.
It requires the benchmarking of the system performance against related works, and the documentation of the results.
The task is required to provide an assessment of the project's success against the performance indicators.


\section{Risk Assessment}
This project is conducted in the low-risk laboratory covered by general OHS laboratory rules, and in a home setting. 
There are no hazardous materials or dangerous equipment used in the project, and the risk of injury is negligible.
The only risk to the project is hardware failure or proprietary software issues, which can be mitigated by using open-source software and hardware, and regular backups of the project files.
Redundancies are also in place for hardware failure, as two FPGA developments boards are designated for use.