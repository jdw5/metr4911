\chapter[Introduction]{Introduction}

\label{Chap:Motivation}

\section{Motivation}

In recent years, the demand for efficient and versatile image processing systems has surged across various domains, including surveillance, medical imaging, autonomous vehicles, and more. 
Field Programmable Gate Arrays (FPGAs) provide a reconfigurable, low-power embedded platform with parallel processing capabilities akin to graphic processing units (GPUs) commonly used to accelerate demanding computing tasks. 
Image processing is a good candidate for such application of parallel processing, due to the increasing algorithm complexity and large volume of data involved. \cite{Efficient}
Hence, FPGAs could provide a suitable platform for real-time image processing tasks, offering a balance of performance, power efficiency, and flexibility.

One significant aspect of image processing systems is the choice of the underlying processor architecture. 
Traditional approaches often rely on general-purpose CPUs or GPUs to execute image processing tasks. 
However, these architectures may not always offer the best balance of performance, power efficiency, and flexibility for image processing applications. 
In recent years, the RISC-V instruction set architecture (ISA) has gained traction as an open, customizable, and energy-efficient alternative to proprietary processor designs.
These soft processors can be implemented on FPGAs, providing a controllable and scalable platform for real-time image processing tasks whilst enabling the processing tasks to be performed in parallel.

This project will focus on the exploration and implementation of image detection algorithms using the RISC-V processor architecture deployed on an FPGA platform. 
By leveraging the configurability of FPGAs and the energy efficiency of the RISC-V ISA, this research aims to develop a scalable and adaptable solution for image processing using FPGA-accelerated hardware.
The data processing capacity is large that the processing speed must be strict to meet the demands of real-time time image transmission \cite{Video}.
Hence, an FPGA system-on-chip (SoC) offers a means for a low power consumption, low latency but high throughput platform - all of which are essential for the increasing computational demands of image processing tasks \cite{Throughput}.

The work will produce a hardware implementation of an image processing system that can be controlled by a RISC-V processor, and demonstrate the benefits of FPGA-based hardware acceleration for image detection tasks.