\chapter[Introduction]{Introduction}
\label{Chap:Intro}

% ***************************************************
% Introduction
% ***************************************************


\section{Motivation}

In recent years, the demand for efficient and versatile image processing systems has surged across various domains, including surveillance, medical imaging, autonomous vehicles, and more. 
Field Programmable Gate Arrays (FPGAs) provide a reconfigurable, low-power embedded platform with parallel processing capabilities akin to graphic processing units (GPUs) commonly used to accelerate demanding computing tasks. 
Image processing is a good candidate for such application of parallel processing, due to the increasing algorithm complexity and large volume of data involved \cite{Efficient}.
These processes have a shared step in the form of convolution, which is a compute-heavy operation often parallelised in software-based implementations.
Hence, FPGAs could provide a suitable platform for real-time image processing tasks, offering a balance of performance, power efficiency, and flexibility.

One significant aspect of image processing systems is the choice of the underlying processor architecture. 
Traditional approaches often rely on general-purpose CPUs or GPUs to execute image processing tasks. 
However, these architectures may not always offer the best balance of performance, power efficiency, and flexibility for image processing applications. 
In recent years, the RISC-V instruction set architecture (ISA) has gained traction as an open, customizable, and energy-efficient alternative to proprietary processor designs.
These soft processors can be implemented on FPGAs, providing a controllable and scalable platform for deferring image processing tasks to dedicated hardware, enabling for parallelisation.

This project will focus on the exploration and implementation of image detection algorithms using the RISC-V processor architecture deployed on an FPGA platform. 
By leveraging the configurability of FPGAs and the energy efficiency of the RISC-V ISA, this research aims to develop a scalable and adaptable solution for handling the core component of image processing tasks using FPGA-accelerated hardware.
The data processing capacity is large that the processing speed must be strict to meet the demands of real-time time image transmission \cite{Video}.
Hence, an FPGA system-on-chip (SoC) offers a means for a low power consumption, low latency but high throughput platform - all of which are essential for the increasing computational demands of image processing tasks \cite{Throughput}.

The work will produce a hardware implementation of an image processing system that can be controlled by a RISC-V processor, and demonstrate the benefits of FPGA-based hardware acceleration for image processing tasks.

\section{Definition}

Traditional software-based image processing techniques on embedded systems lack the parallelisation of larger distributed systems which use GPU architecture to process images.
This project aims to provide a robust and extensible platform for image processing tasks which can be used as a building block for more complex image processing pipelines on an FPGA platform.
By allowing for extensibility of the image processing pipeline, the project aims to provide a flexible platform for future research into the use of FPGAs in the field of image processing.

This projects aims to:
\begin{itemize}
    \item Demonstrate the practicality and benefits of FPGA-based hardware acceleration for image processing tasks.
    \item Illustrate the practicality of neural network layers being applied to resource-limited devices
    \item Demonstrate the control of an image processing pipeline using a RISC-V softcore processor.
    \item Evaluate the performance, efficiency, and scalability of a hardware-implemented image processing system.
\end{itemize}

\section{Scope}

A scope has been defined to better target the aims of this thesis, and exclude any work which is deemed to not be primary to the aims.
The thesis is focused on implementing a convolutional core, and by extension a convolutional neural network, to be used as a hardware accelerator for image processing tasks, and interfaced to with a RISC-V processor.
The scopes of this thesis are hence defined as:

\begin{itemize}
    \item Development of a hardware-implemented convolutional core to accelerate image processing tasks.
    \item Extension of this core to implement a convolutional neural network.
    \item Configuration of neural network layers to be mapped onto the FPGA fabric.
\end{itemize}

There are a number of components which are outside of the scope of this thesis due to the broadness of image processing techniques:
\begin{itemize}
    \item Implementing of image processing techniques which use convolution but extend beyond it
    \item Training of a CNN on FPGA
    \item Finetuning of a CNNs
\end{itemize}